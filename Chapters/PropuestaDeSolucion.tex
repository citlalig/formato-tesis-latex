\chapter{Propuesta de Soluci�n} 

\section{Selecci�n de un Conjunto de Datos}

Hasta Abril de 2015 se reportan un total de 358 conjuntos de datos disponibles en el sitio web \texorpdfstring{\href{http://catalogo.datos.gob.mx/dataset}
                {http://catalogo.datos.gob.mx/dataset}}
                {http://catalogo.datos.gob.mx/dataset}

El tama�o de la muestra seleccionada para este estudio fue calculado con base en los procedimientos sugeridos en la teor�a del muestreo y probabilidad. Las variables a considerar son las siguientes: \\

\begin{table}[!hbt]
\begin{center}
\begin{tabular}{|l|l|}
\hline
Variable & Descripci�n\\
\hline
n & Tama�o de la muestra\\
N & Tama�o del universo\\
p & Probabilidad de ocurrencia\\
q & Probabilidad de no ocurrencia\\
Me & Margen de error o precisi�n\\
Nc & Nivel de confianza o exactitud\\
\hline
\end{tabular}
\caption{Variables para f�rmula de c�lculo de el tama�o de muestra}
\end{center}
\end{table}

\[n = \frac{Npq}{\left[\frac{Me^2}{Nc^2} \left(N - 1\right) \right] + PQ} \]

\clearpage

\begin{table}[!hbt]
\begin{center}
\begin{tabular}{|l|l|}
\hline
Variable & Descripci�n\\
\hline
N & 358\\
p & .5 (Mayor punto de incertidumbre 50\%)\\
q & 1 - .5 = .5\\
Me & 5 (Margen de error o precisi�n +/- 5 \%)\\
Nc & 1.96 (95 \% de nivel de confianza o exactitud)\\
\hline
\end{tabular}
\caption{Valores reales para f�rmula de c�lculo de el tama�o de muestra}
\end{center}
\end{table}


Al substituir la f�rmula obtenemos un tama�o de muestra recomendado de: \textbf{186}\\

La distribuci�n de los documentos de la muestra, organizados por tipo de formato e instituci�n que lo genera, fue seleccionado de acuerdo a la disponibilidad de los mismos y se distribuye de la siguiente manera:\\

\begin{table}[!hbt]
\begin{center}
\begin{tabular}{|l|l|l|}
\hline
Cantidad & Formato & Instituciones\\
\hline
CSV & 89 & PEMEX, SAGARPA\\
XML & 55 & PEMEX\\
JSON & 3 & PROMEXICO\\
XLS & 2 & SHCP\\
TXT & 37 & SEDESOL\\
\hline
\end{tabular}
\caption{Selecci�n de tama�o de muestra por tipo de formato}
\end{center}
\end{table}

\clearpage
\section{Recuperaci�n de datos usando la API de CKAN}


