\chapter{Resultados Esperados} 

Como producto de el desarrollo de esta tesis y su proyecto de intervenci�n se espera obtener una aplicaci�n que aporte novedades en el �mbito de sistemas de b�squeda orientado a archivos utilizando un m�todo de b�squeda avanzada basado en Apache Lucene  y empleando como fuente de informaci�n los datos abiertos liberados por Instituciones Gubernamentales. \\
Dado que la aplicaci�n estar� desarrollada en su totalidad con herramientas de c�digo abierto y dise�ada para alimentarse con informaci�n de datos abiertos, es natural que la aplicaci�n misma tambi�n sea distribuida libremente por lo que se espera que la implementaci�n y el c�digo fuente este disponible en un repositorio p�blico al acceso de cualquier persona.\\
As� mismo, se espera que la aplicaci�n pueda ser publicada en un servidor p�blico y gratuito para que este disponible a la ciudadan�a en general.\\
Los requerimientos b�sicos para la implementaci�n de la aplicaci�n comprenden:
\begin{itemize}
\item La navegaci�n en la plataforma deber� ser sencilla e intuitiva para cualquier usuario.
\item Los formatos en que los datos son procesados deber� ser uniforme. 
\item El motor de b�squeda deber� permitir consultas tanto por metadatos, etiquetas y al propio contenido de un documento.
\end{itemize}
Alcanzar estos objetivos contribuir� a un cambio cultural al construir la capacidad de comprensi�n de la informaci�n, as� como la habilidad de encontrar, utilizar y compartir datos y alentar la colaboraci�n entre gobiernos, negocios y sociedad civil para combatir los principales retos de la actualidad.  