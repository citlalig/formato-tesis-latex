\chapter{Anexo - Aplicaci�n de B�squeda (C�digo Fuente)}
\label{SearchApp}

\begin{figure}[h!]
\begin{minted}[frame=single,
               framesep=3mm,
               xleftmargin=21pt,
               tabsize=4]{java}    
package datos.gob.mx;
import ...

/**
 * This code was originally written for Erik's Lucene intro java.net
 * article and modified by Citlali Garcia 
 * to meet the application needs.
 */
 
public class Indexer {
	public static void main(String[] args) throws Exception {
		if (args.length != 2) {
			throw new IllegalArgumentException("Usage: java "
					+ Indexer.class.getName() 
					+ " <index dir> <data dir>");
		}
		String indexDir = args[0];
		String dataDir = args[1];
		long start = System.currentTimeMillis();
		Indexer indexer = new Indexer(indexDir);
		int numIndexed;
		try {
			numIndexed = 
				indexer.index(dataDir, new TextFilesFilter());
		} finally {	indexer.close();}
		long end = System.currentTimeMillis();
		System.out.println("Indexing " + numIndexed + " files took "
				+ (end - start) + " milliseconds");
	}       
\end{minted}
\caption{C�digo Fuente para clase Indexer.java (Parte 1)} 
\end{figure}


\begin{figure}[h!]
\begin{minted}[frame=single,
               framesep=3mm,
               xleftmargin=21pt,
               tabsize=4]{java}    
	private IndexWriter writer;
	public Indexer(String indexDir) throws IOException {
		Directory dir = FSDirectory.open(new File(indexDir));
		Analyzer analyzer = new StandardAnalyzer(Version.LUCENE_4_9);
		IndexWriterConfig iwc = 
			new IndexWriterConfig(Version.LUCENE_4_9, analyzer);
		writer = new IndexWriter(dir, iwc);
	}

	public void close() throws IOException { writer.close(); }

	public int index(String dataDir, FileFilter filter) 
		throws Exception {
		File[] files = new File(dataDir).listFiles();
		for (File f : files) {
			if (!f.isDirectory() && !f.isHidden() && f.exists() 
				&& f.canRead() 
				&& (filter == null || filter.accept(f))) {
					indexFile(f);
			}
		}
		return writer.numDocs();
	}

	private static class TextFilesFilter implements FileFilter {
		public boolean accept(File path) {
			return path.getName().toLowerCase().endsWith(".txt");
		}
	}

	protected Document getDocument(File f) throws Exception {
		Document doc = new Document();
		doc.add(new StringField("fullpath", f.getPath(), 
					Field.Store.YES));
		doc.add(new TextField("contents", new BufferedReader(
				new InputStreamReader(new FileInputStream(f),
						StandardCharsets.UTF_8))));
		doc.add(new StringField("filename", f.getName(), 
			Field.Store.YES));
		doc.add(new LongField("modified", f.lastModified(), 
			Field.Store.NO));
		return doc;
	}

	private void indexFile(File f) throws Exception {
		System.out.println("Indexing " + f.getCanonicalPath());
		Document doc = getDocument(f);
		writer.addDocument(doc);
	}
}    
\end{minted}
\caption{C�digo Fuente para clase Indexer.java (Parte 2)} 
\end{figure}


\begin{figure}[h!]
\begin{minted}[frame=single,
               framesep=3mm,
               xleftmargin=21pt,
               tabsize=4]{java}    
package datos.gob.mx;
import ...
public class TikaIndexer extends Indexer {

	static Set<String> textualMetadataFields = new HashSet<String>();
	static {
		textualMetadataFields.add(
			TikaCoreProperties.TITLE.getName());
		textualMetadataFields.add(
			TikaCoreProperties.COMMENTS.getName());
		textualMetadataFields.add(
			TikaCoreProperties.KEYWORDS.getName());
		textualMetadataFields.add(
			TikaCoreProperties.DESCRIPTION.getName());
		textualMetadataFields.add(
			TikaCoreProperties.KEYWORDS.getName());
	}

	public static void main(String[] args) throws Exception {
		if (args.length != 2) {
			throw new IllegalArgumentException("Usage: java "
					+ TikaIndexer.class.getName() 
					+ " <index dir> <data dir>");
		}

		TikaConfig config = TikaConfig.getDefaultConfig();
		System.out.println("Mime type parser:" + config.getParser());

		String indexDir = args[0];
		String dataDir = args[1];

		long start = new Date().getTime();
		TikaIndexer indexer = new TikaIndexer(indexDir);
		int numIndexed = indexer.index(dataDir, null);
		indexer.close();
		long end = new Date().getTime();

		System.out.println("Indexing " + numIndexed + " files took "
				+ (end - start) + " milliseconds");
	}
\end{minted}
\caption{C�digo Fuente para clase TikaIndexer.java (Parte 1)} 
\end{figure}


\begin{figure}[h!]
\begin{minted}[frame=single,
               framesep=3mm,
               xleftmargin=21pt,
               tabsize=4]{java}    
public TikaIndexer(String indexDir) throws IOException {
		super(indexDir);
	}

	protected Document getDocument(File f) throws Exception {
		Metadata metadata = new Metadata();
		metadata.set(Metadata.RESOURCE_NAME_KEY, f.getName());
		InputStream is = new FileInputStream(f);
		Parser parser = new AutoDetectParser();
		ContentHandler handler = new BodyContentHandler();
		ParseContext context = new ParseContext();
		context.set(Parser.class, parser);
		try {
			parser.parse(is, handler, metadata, new ParseContext());
		} finally {
			is.close();
		}
		Document doc = new Document();
		doc.add(new StringField("fullpath", f.getPath(), 
			Field.Store.YES));
		doc.add(new StringField("filename", f.getCanonicalPath(),
				Field.Store.YES));
		doc.add(new LongField("modified", f.lastModified(), 
			Field.Store.NO));
		doc.add(new TextField("contents", new BufferedReader(
				new InputStreamReader(new FileInputStream(f),
						StandardCharsets.UTF_8))));
		doc.add(new TextField("contents", handler.toString(), 
			Field.Store.NO));
		for (String name : metadata.names()) {
			String value = metadata.get(name);
			if (textualMetadataFields.contains(name)) {
				doc.add(new TextField("contents", value,
					Field.Store.NO));
			}
			doc.add(new TextField(name, value, 
				Field.Store.YES));
		}
		return doc;
	}
}
\end{minted}
\caption{C�digo Fuente para clase TikaIndexer.java (Parte 2)} 
\end{figure}


\begin{figure}[h!]
\begin{minted}[frame=single,
               framesep=3mm,
               xleftmargin=21pt,
               tabsize=4]{java}    
package datos.gob.mx;

import org.apache.lucene.document.Document;
import org.apache.lucene.index.DirectoryReader;
import org.apache.lucene.index.IndexReader;
import org.apache.lucene.search.IndexSearcher;
import org.apache.lucene.search.Query;
import org.apache.lucene.search.ScoreDoc;
import org.apache.lucene.search.TopDocs;
import org.apache.lucene.store.Directory;
import org.apache.lucene.store.FSDirectory;
import org.apache.lucene.queryparser.classic.QueryParser;
import org.apache.lucene.queryparser.classic.ParseException;
import org.apache.lucene.analysis.Analyzer;
import org.apache.lucene.analysis.standard.StandardAnalyzer;
import org.apache.lucene.util.Version;

import java.io.File;
import java.io.IOException;

/**
 * This code was originally written for Erik's Lucene intro java.net 
 * article and modified by Citlali Garcia to meet 
 * the application needs.
 */
public class Searcher {

	public static void main(String[] args) throws 
		IllegalArgumentException, IOException, ParseException {
		if (args.length != 2) {
			throw new IllegalArgumentException("Usage: java "
					+ Searcher.class.getName() 
					+ " <index dir> <query>");
		}

		String indexDir = args[0];
		String query = args[1];
		search(indexDir, query);
	}
	     
\end{minted}
\caption{C�digo Fuente para clase Searcher.java (Parte 1)} 
\end{figure}


\begin{figure}[h!]
\begin{minted}[frame=single,
               framesep=3mm,
               xleftmargin=21pt,
               tabsize=4]{java}    
  public static void search(String indexDir, String q) 
  	throws IOException, ParseException {

		Directory dir = FSDirectory.open(new File(indexDir));
		IndexReader reader = DirectoryReader.open(dir);
		IndexSearcher is = new IndexSearcher(reader);

		Analyzer analyzer = new StandardAnalyzer(Version.LUCENE_4_9);
		QueryParser parser = new QueryParser(Version.LUCENE_4_9, 
			"contents", analyzer);

		Query query = parser.parse(q);
		long start = System.currentTimeMillis();
		TopDocs hits = is.search(query, 10);
		long end = System.currentTimeMillis();

		System.err.println("Found " + hits.totalHits 
				+ " document(s) (in " + (end - start) 
				+ " milliseconds) that matched query '" + q
				+ "':");

		for (ScoreDoc scoreDoc : hits.scoreDocs) {
			Document doc = is.doc(scoreDoc.doc);
			System.out.println(doc.get("fullpath"));
		}
		reader.close();
	}
}
\end{minted}
\caption{C�digo Fuente para clase Searcher.java (Parte 2)} 
\end{figure}
