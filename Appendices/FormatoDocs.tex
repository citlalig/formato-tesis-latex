\chapter{Anexo - Resumen de formatos de archivos m�s comunes}


\textbf{JSON} \\  Acr�nimo de JavaScript Object Notation, es un formato ligero para el intercambio de datos. JSON es un subconjunto de la notaci�n literal de objetos de JavaScript que no requiere el uso de XML.\\
\textbf{XML} \\ Siglas en ingl�s de eXtensible Markup Language ('lenguaje de marcas extensible'), es un lenguaje de marcas desarrollado por el World Wide Web Consortium (W3C) utilizado para almacenar datos en forma legible. \\
\textbf{XLS} \\ La extensi�n de archivo por defecto para aplicaciones de hojas de c�lculo. \\
\textbf{CSV} \\ Tipo de documento en formato abierto sencillo para representar datos en forma de tabla, en las que las columnas se separan por comas y las filas por saltos de l�nea.  \\
\textbf{TXT} \\La extensi�n para archivo compuesto �nicamente por texto sin formato, s�lo caracteres, lo que lo hace tambi�n legible por humanos.  \\
\textbf{ZIP} \\Formato de compresi�n sin p�rdida, muy utilizado para la compresi�n de datos como documentos, im�genes o programas. \\
\textbf{KMZ} \\ Es un lenguaje de marcado basado en XML para representar datos geogr�ficos en tres dimensiones en formato comprimido.  \\